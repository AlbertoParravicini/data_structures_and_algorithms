\documentclass[14pt]{beamer}

% Presento style file
\usepackage{config/presento}

% custom command and packages
% custom packages
\usepackage{textpos}
\setlength{\TPHorizModule}{1cm}
\setlength{\TPVertModule}{1cm}

\newcommand\crule[1][black]{\textcolor{#1}{\rule{2cm}{2cm}}}



% Information
\title{CONVEX HULLS}
\subtitle{Chan's optimal output sensitive algorithms for convex hulls}
\author{Alberto Parravicini}
\institute{Université libre de Bruxelles}
\date{December 15, 2016}

\setbeamertemplate{bibliography item}[text] % No icons in bibliography

\begin{document}

% Title page
\begin{frame}[plain]
\maketitle
\end{frame}

\begin{frame}{What's on the menu?}
 \begin{fullpageitemize}
  \item<1->[\rtarrow] {\montserratfont\ Basic notions}
  \item<2->[\rtarrow] {\montserratfont\ Algorithms for convex hulls}
  \item<3->[\rtarrow] {\montserratfont\ Optimal 2D algorithm}
  \item<4->[\rtarrow] {\montserratfont\ Optimal 3D algorithm}
 \end{fullpageitemize}
\end{frame}

\section{Basic Notions}
\begin{frame}{Crash course on convex hulls}
 \begin{center}
 A \textit{convex set} $S$ is a set in which, $\forall\ x, y \in S$, the segment $xy \subseteq S$.
 \vspace{8mm}
 	
 Given a set of points $P$ in $d$ dimensions, the \textbf{Convex Hull} $\mathbf{CH(P)}$ of $P$ is:
 
 \begin{baseitemize}
 	\item<2->[\rtarrow] the \textit{minimal convex set} containing $P$.
 	\vspace{3mm}
 	\item<3->[\rtarrow] the union of \textit{all convex combinations} of  points in $P$, i.e. the points $CH(P)$ are s.t.
	\vspace{-2mm}
 	$$\sum_{i=1}^{|P|}{w_i \cdot x_i},\ \forall x_i \in P,\ \forall w_i:\ w_i \geq 0\ and\ \sum_{i=1}^{|P|}{w_i}=1$$
 \end{baseitemize}
 \end{center}
\end{frame}

\begin{frame}{Output sensitive algorithm}
	\begin{center}
	The complexity of an \textcolor{colorblue}{\textit{output-sensitive}} algorithm is a function of both the \textcolor{colorblue}{\textit{input size}} and the \textcolor{colorblue}{\textit{output size}}.
	\end{center}
\end{frame}


\section{Algorithms for convex hulls}

% JARVIS MARCH %%%%%%%%%%%%%%%%%%%%%%%
\begin{frame}{Jarvis's march}
	\begin{center}
		\begin{fullpageitemize}		
			\item<1->[\rtarrow] \textbf{Core idea:} given an edge $\bm{pq}$ of the convex hull, the next edge $\bm{qr}$ to be added will maximize the angle $\bm{\angle pqr}$\\
			\vspace{8mm}
			
			\item<2->[\rtarrow] At each step we scan all the $\bm{n}$ points.\\
			\item<3->[\rtarrow] How many steps? $\bm{h}$, size of the hull.\\
			\item<4->[\rtarrow] \textbf{Complexity:} \textcolor{colorgreen}{$\bm{O(n h)}$}
		\end{fullpageitemize}
	\end{center}
\end{frame}

\begin{frame}{One step, visually}
	\begin{figure}[H]
		\centering
		\includegraphics[width=0.8\textwidth, keepaspectratio=1]{{"../Latex/jarvis_march"}.pdf}
		\caption{\emph{A step of the \textit{Jarvis March}. The \textcolor{colorred}{red} line is the next segment that will be added to the hull.}}
	\end{figure}
\end{frame}

\begin{frame}{Jarvis's march}
	\fontsize{10}{12}\notosansfont
	\begin{algorithm}[H]
		\SetKwInput{Input}{Input}
		\SetKwInput{Output}{Output}
		\SetKwFor{Loop}{Loop}{}{EndLoop}
		\DontPrintSemicolon
		\SetAlgoVlined
		\Input{a list $S$ of bidimensional points.}
		\Output{the convex hull of the set, sorted counterclockwise.}
		hull =[]\\
		$x_0 =$ the leftmost point.\\
		hull.push($x_0$)\\
		\Loop{hull.last() != hull.first()}{
			candidate = S.first()\\
			\ForEach{p in S}{
				\If{p != hull.last() and p is on the right of the segment "hull.last(), candidate"}{
					candidate = p}}
			\leIf{candidate != hull.first}
			{hull.push(candidate)}{break}
		}
		return hull
		\caption{Jarvis March}
	\end{algorithm}

\end{frame}

% GRAHAM SCAN %%%%%%%%%%%%%%%%%%%%%%%
\begin{frame}{Graham's scan}
	\begin{center}
		\vspace{-5mm}
		\begin{fullpageitemize}
			\item<1->[\rtarrow] Sort the points in counter-clockwise order around the leftmost point - \textbf{Cost:}  \textcolor{colorgreen}{$\bm{O(n\ log\ n)}$}
			\vspace{5mm}
			\item<2->[\rtarrow] Keep the current hull in a stack $H = \{h_0,\ \ldots,\ h_{curr}\}$.
			\vspace{5mm}
			
			\item<3->[\rtarrow] Inspect each point $p$ in order - \textbf{Cost:}  \textcolor{colorgreen}{$\bm{O(n)}$}
			
			\vspace{2mm}
			\fontsize{12}{14}\notosansfont
			\begin{sublist}
				\item<4->While $h_{curr-1}h_{curr}p$ is a right turn, pop $h_{curr}$.
				\item<5->Push $p$ to $H$.
			\end{sublist}
		
			\vspace{5mm}
			\item<6->[\rtarrow] \textbf{Overall complexity:} \textcolor{colorgreen}{$\bm{O(n\ log\ n)}$}
		\end{fullpageitemize}	
	\end{center}
\end{frame}

\begin{frame}{Graham's scan}
	\fontsize{10}{12}\notosansfont
	\begin{algorithm}[H]
		\SetKwInput{Input}{Input}
		\SetKwInput{Output}{Output}
		\SetKwFor{Loop}{Loop}{}{EndLoop}
		\DontPrintSemicolon
		\SetAlgoVlined
		\Input{a list $S$ of bidimensional points.}
		\Output{the convex hull of the set, sorted counterclockwise.}
		hull =[]\\
		$x_0 =$ the leftmost point.\\
		Put $x_0$ as the first element of $S$.\\
		Sort the remaining points in counter-clockwise order, with respect to $x_0$.\\
		Add the first 3 points in $S$ to the hull. \\
		
		
		\ForAll{the remaining points in S}{
			\While{hull.second\_to\_last(), hull.last(), p form a right turn}{
				hull.pop()
			}
			hull.push(p)
		}
		return hull
		\caption{Graham Scan}
	\end{algorithm}	
\end{frame}


% CHAN 2D %%%%%%%%%%%%%%%%%%%%%%%
\begin{frame}{Can we do better?}
	\begin{center}
		\begin{fullpageitemize}
			\item<1-> So far:
			\begin{baseitemize}
				\item<2->[\rtarrow] \textbf{Jarvis's march:} \textcolor{colorgreen}{$\bm{O(nh)}$}
				\item<3->[\rtarrow] \textbf{Graham's scan:} \textcolor{colorgreen}{$\bm{O(n\ log\ n)}$}
			\end{baseitemize}
		\vspace{5mm}		
		\item<4->[\rtarrow] How do we compare their complexity?\\
		\item<5->[\rtarrow] We would like to do even better!
		\end{fullpageitemize}
		\onslide<6->Can we get to \textcolor{colorgreen}{$\bm{O(n\ log\ h)}$}?
	\end{center}
\end{frame}

\section{Chan's 2D algorithm}
\begin{frame}{Chan's 2D algorithm}
	\begin{center}
		\vspace{-2mm}
		\begin{fullpageitemize}
			\item<1->[\rtarrow] \textbf{Idea:} Some points will never be in the hull! Discard them to speed up \textit{Jarvis's march}.
			%\vspace{-5mm}
			
			\item<2->[\rtarrow] Combine \textit{Jarvis's march} and \textit{Graham's scan}.
			
			\vspace{4mm}
			\item<3->[\rtarrow] Algorithm (\textbf{Chan 2D}):			
			\fontsize{12}{14}\notosansfont
			\begin{sublist}
				\item<3-> Split the $n$ points in groups of size $m$ \\($\lceil n/m \rceil$ groups).
				\item<4-> Compute the hull $H_i$ of each group \\ \textbf{Cost:} \textcolor{colorgreen}{$\bm{O((n/m) \cdot (m\ log\ m)) = O(n\ log\ m)}$}
				
				\item<5-> Until the hull is complete, repeat:
				\begin{sublist}
					\item<6-> Given the current hull $H_{tot} = \{h_0,\ \ldots,\ h_{curr}\}$, find for each $H_i$ the point right tangent to $h_{curr}$ (binary search, \textcolor{colorgreen}{$\bm{O(log\ m) \lceil n/m \rceil}$}).	
					\item<7-> Pick the tangent point $p$ that maximizes  $\angle h_{curr-1}h_{curr}p$.
				\end{sublist}
			\end{sublist}
		\end{fullpageitemize}	
	\end{center}
\end{frame}

\begin{frame}{One step, visually}
	\begin{figure}[H]
		\centering
		\includegraphics[width=0.8\textwidth, keepaspectratio=1]{{"../Latex/chan_step"}.pdf}
		\caption{\emph{A step of \textit{Chan's algorithm}. In \textcolor{colorblue}{blue}, the existing hull, in \textcolor{colororange}{orange}, the tangents, in \textcolor{colorred}{red}, the new edge that will be added.}}
	\end{figure}
\end{frame}

\begin{frame}{Chan's 2D algorithm, part 2}
	\begin{center}
		\vspace{-5mm}
		\begin{fullpageitemize}
			\item<1-> \textbf{Complexity:} \textcolor{colorgreen}{$\bm{O(n\ log\ m + (h(n/m)log\ m))}$}
			\item<2-> What about $m$? Let's pretend the size $h$ of the final hull is know.
			\item<3-> With $m = h$, we get complexity $O(n\ log\ h + (h(n/h)log\ h)) =$ \textcolor{colorgreen}{$\bm{O(n\ log\ h)}$}
			\item<4-> But h is not known! 
			
			\fontsize{12}{14}\notosansfont
			\begin{sublist}
				\item<5-> Reiterate the algorithm, with $m = 2^{2^{i}} = H$
				\item<6-> Iteration cost: \textcolor{colorgreen}{$\bm{O(n\ log\ H) = O(n2^i)}$}
				\item<7-> $O\left(\sum_{i = 1}^{\lceil log\ log\ h\rceil}{n2^i}\right) = O(n2^{\lceil log\ log\ h\rceil + 1}) =$ \textcolor{colorgreen}{$\bm{O(n\ log\ h)}$}
			\end{sublist}
			
			
		\end{fullpageitemize}
	\end{center}
\end{frame}

\begin{frame}{Chan's 2D pseudo-code}
	\fontsize{8}{10}\notosansfont
	\begin{algorithm}[H]
		\SetKwInput{Input}{Input}
		\SetKwInput{Output}{Output}
		\SetKwFor{Loop}{Loop}{}{EndLoop}
		\DontPrintSemicolon
		\SetAlgoVlined
		\Input{a list $S$ of bidimensional points, the parameters $m, H$}
		\Output{the convex hull of the set, sorted counterclockwise, or an empty list, if $H$ is $<$ $h$}
		Partition $S$ into subsets $S_1,\ \ldots,\ S_{\lceil n/m \rceil}$.\\
		\For{$i = 1,\ \ldots,\  \lceil n/m \rceil$}
		{
			Compute the convex hull of $S_i$ by using Graham Scan, store the output in a counter-clockwise sorted list.
		}  
		$p_0 = (0, -\infty)$\\
		$p_1 =$ the leftmost point of $S$.
		
		\For{$k = 1,\ \ldots, \ H$}
		{
			\For{$i = 1,\ \ldots,\ \lceil n/m \rceil$}
			{
				Compute the points $q_i \in S$ that maximizes $\angle p_{k-1}p_kq_i$, with $q_i \ne p_k$, by performing binary search on the vertices of the partial hull $S_i$.
			}
			$p_{k+1} = $ the point $q \in \{q_1,\ \ldots,\ q_{\lceil n/m \rceil}\}$.\\
			\lIf{$p_{k+1} = p_t$}{return $\{p_1,\ \ldots,\ p_k\}$}
		}
		
		return \textit{incomplete}
		\caption{ChanHullStep, a step of Chan's algorithm}
	\end{algorithm}
\end{frame}

\begin{frame}{Chan's 2D pseudo-code, reprise}
	\fontsize{8}{10}\notosansfont
	\begin{algorithm}[H]
		\SetKwInput{Input}{Input}
		\SetKwInput{Output}{Output}
		\SetKwFor{Loop}{Loop}{}{EndLoop}
		\DontPrintSemicolon
		\SetAlgoVlined
		\Input{a list $S$ of bidimensional points}
		\Output{the convex hull of the set}
		\For{$i = 1,2,\ \ldots$}{
			L = ChanHullStep(S, m, H), where $m = H = min\{|S|, 2^{2^i}\}$\\
			\lIf{L $\ne$ incomplete}{return L}
		}
		\caption{Chan's algorithm}
	\end{algorithm}
\end{frame}

\begin{frame}{Chan's 2D - Empirical analysis}
	\fontsize{12}{14}\notosansfont
	\begin{fullpageitemize}
		\item Is a real implementation \textcolor{colorgreen}{$\bm{O(n\ log\ h)}$}?
		\vspace{2mm}
		\item \textbf{Test 1:} fixed hull size (1000), increasing number of points.
	\end{fullpageitemize}
	\vspace{-9mm}
	\begin{figure}[H]
		\centering
		\includegraphics[width=1\textwidth, keepaspectratio=1]{{"../Latex/increasing_points_pres"}.pdf}
	\end{figure}	
\end{frame}

\begin{frame}{Chan's 2D - Empirical analysis}
	\fontsize{12}{14}\notosansfont
	\begin{fullpageitemize}
		\item Is a real implementation \textcolor{colorgreen}{$\bm{O(n\ log\ h)}$}?
		\vspace{2mm}
		\item \textbf{Test 2:} fixed number of points (40000), increasing hull size.
	\end{fullpageitemize}
	\vspace{-9mm}
	\begin{figure}[H]
		\centering
		\includegraphics[width=1\textwidth, keepaspectratio=1]{{"../Latex/increasing_hull_pres"}.pdf}
	\end{figure}	
\end{frame}



% CHAN 3D %%%%%%%%%%%%%%%%%%%%%%%
\section{Chan's 3D algorithm}
\begin{frame}{Chan's 3D algorithm}
	\begin{center}
		\vspace{-2mm}
		\begin{fullpageitemize}
			\item<1->[\rtarrow] \textbf{Idea:} The structure of the algorithm doesn't change, just the ingredients.
			
			\fontsize{12}{14}\notosansfont
			\begin{sublist}
				\vspace{2mm}
				\item<2->[] Given a set $S$ of $n$ points with an hull of size $h$:
				\vspace{2mm}
				\item<3-> Jarvis's march $\longrightarrow$ \textbf{Chand and Kapur's gift wrapping} - \textcolor{colorgreen}{$\bm{O(nh)}$}.
				\item<4-> Graham's scan $\longrightarrow$ \textbf{Preparata and Hong 3D Algorithm} - \textcolor{colorgreen}{$\bm{O(n\ log\ n)}$}.
				\item<5-> Finding tangents in 2D $\longrightarrow$ \textbf{Supporting planes in 3D with Dobkin-Kirkpatrick hierarchy} - \textcolor{colorgreen}{$\bm{O(log\ n)}$}.
			\end{sublist}		
			
			\item<6-> The overall complexity is still \textcolor{colorgreen}{$\bm{O(n\ log\ h)}$}.
		\end{fullpageitemize}	
	\end{center}
\end{frame}

\begin{frame}{Chan's 3D algorithm}
	\begin{center}
		\begin{fullpageitemize}
			\item<1->[\rtarrow] \textbf{3D Gift wrapping}
			\fontsize{12}{14}\notosansfont
			\begin{sublist}
				\item<2-> Pick an edge $e_j$ of a face $f$ of the partial hull. 
				\item<3-> Find $p$ that maximizes the angle between $f$ and the new face given by $e_j$ and $p$. Repeat for the 3 edges of $f$.
				\item<4-> Use \textit{breadth first visit} to build the entire hull.
			\end{sublist}
			
			\item<5->[\rtarrow] \textbf{Preparata and Hong 3D algorithm, a divide-and-conquer algorithm for convex hulls}	
			\fontsize{12}{14}\notosansfont
			\begin{sublist}
				\item<6-> Split the set of points $S$ in $S_1$ and $S_2$.
				\item<7-> Recursively split until the base case, then build the convex hull.
				\item<8-> Merge the partial hulls.
			\end{sublist}
		\end{fullpageitemize}
	\end{center}
\end{frame}

\begin{frame}{Chan's 3D algorithm}
	\begin{center}
		\begin{fullpageitemize}
			\vspace{-4mm}
			\item<1->[\rtarrow] To find supporting planes, store each partial hull as a \textbf{Dobkin-Kirkpatrick hierarchy}
			\fontsize{12}{14}\notosansfont
			\begin{sublist}
				\item<2-> Sequence $P_0,\ \ldots,\ P_k$ of increasingly smaller approximations of a polyhedron $P = P_0$.
				\item<3-> Build $P_{i+1}$ from $P_i$ by picking a \textcolor{colorblue}{maximal set of independent vertices} of $P_i$. 
				\item<4-> Overall, building the hierarchies takes \textcolor{colorgreen}{$\bm{O(n)}$}.
			\end{sublist}
			
			\vspace{4mm}
			\item<5->[\rtarrow] \textbf{Finding supporting planes in 3D}	
			\fontsize{12}{14}\notosansfont
			\begin{sublist}
				\item<6-> \textbf{Goal:} given an edge $e_j$ and the DK hierarchy of a partial hull $H_i$, find the \textcolor{colorblue}{plane} passing through $e_j$ and \textcolor{colorblue}{tangent} to $H_i$ in $p_t$.
				\item<7-> Find $p_t$ in constant time in $P_k$
				\item<8-> Step up in the hierarchy: if $p_t$ changes, it will move to a \textcolor{colorblue}{neighbour} (constant time check).
				\item<9-> The DK hierarchy has \textcolor{colorgreen}{$\bm{O(log\ m)}$} height.
			\end{sublist}
		\end{fullpageitemize}
	\end{center}
\end{frame}

%find the tangent of the innermost polyedorn in const time
%step up the hierarchy. the new tangent point if different, will be in one of the neighbours of the previous point
%logn step necessary

\framecard[colorgreen]{{\color{white}\hugetext{THANK YOU!}}}

\begin{frame}[allowframebreaks]
	\frametitle{References}
	\nocite{*}
	\scriptsize{\bibliographystyle{plainurl}}	
	\bibliography{../Latex/bibliography} %bibtex file name without .bib extension
	\begin{baseitemize}
		\item \textbf{Beamer theme:} \textit{Presento}, by 
		Ratul Saha. \textit{The research system in Germany}, by Hazem Alsaied
	\end{baseitemize}
\end{frame}


\end{document}